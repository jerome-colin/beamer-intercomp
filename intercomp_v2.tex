% Copyright 2019 by Jordi Inglada
%
% This file may be distributed and/or modified
%
% 1. under the LaTeX Project Public License and/or
% 2. under the GNU Public License.
%

\documentclass[8pt]{beamer}
\usetheme{Cesbio}
\setbeamertemplate{footnote}{%
  \parindent 0em\noindent%
  \raggedright%\raggedright
  \usebeamercolor{footnote}\hbox to 3.5em{\hfil\insertfootnotemark}\insertfootnotetext\par%
}
\setbeamertemplate{navigation symbols}{}
\usepackage[english]{babel}
\usepackage[utf8]{inputenc}
\usepackage[T1]{fontenc}
\usepackage{graphicx}
\usepackage{FiraSans}
\usepackage{mathtools}
\usepackage{ulem}

% The short title appears at the bottom of every slide, the full title is only on the title page
\title{MAJA and MAQT cross-comparison notebook [on-going]} 
\newcommand{\shorttitle}{MAVA vs MAQT notes}

\author{J. Colin\inst{1}}% \and Jean Bombeur\inst{2} \and Igor Le Tractor\inst{3}}
\newcommand{\shortauthor}{J. Colin} % Name to display in the footline
\institute{\inst{1}Centre d'\'Etudes Spatiales de la Biosph\`ere, France \url{jerome.colin@cesbio.cnes.fr}\\
%\inst{2}Université des \^Iles Sandwich, Royaume-Uni\\
%\inst{3}Laboratoire de mécanique agricole, Moldavie
}

\date{\today} % Date, can be changed to a custom date

\begin{document}
{
	\setbeamertemplate{footline}{} % Remove footline in titlepage
	\begin{frame}
		\vspace{.8cm}
		\includegraphics[width=1.5cm]{logo_cesbio_transp.pdf}
	\titlepage % Print the title page as the first slide
	\end{frame}
}

%----------------------------------------------------------------------------------------
\begin{frame}
	\frametitle{Overview}
	\tableofcontents
\end{frame}

%----------------------------------------------------------------------------------------
%   PRESENTATION SLIDES
%----------------------------------------------------------------------------------------

%------------------------------------------------
\section{Experimental setup} 
%------------------------------------------------

\subsection{Area of Interest} 

\begin{frame}
\frametitle{Area of Interest}
	
	TODO: check DTM file, $\Delta_{altitude}$, Water fraction, etc...	
	
	\begin{columns}
		\begin{column}{0.5\textwidth}
			\begin{center}
				31TFJ 2018-03-14(S2B)
		     	\includegraphics[width=1\textwidth]{figs/20180314_S2B.jpg}
		     	13\% clouds
		    \end{center}		
		\end{column}
		\begin{column}{0.5\textwidth}
			\begin{center}
				31TFJ 2018-03-19(S2A)
		     	\includegraphics[width=1\textwidth]{figs/20180319_S2A.jpg}
		     	34\% clouds
		    \end{center}		
		\end{column}
	\end{columns}
\end{frame}

%------------------------------------------------
\subsection{Code configurations} 

\begin{frame}
\frametitle{Codes and GIPP versions}

	Versions of the code:
	
	\begin{itemize}
		\item MAJA refers here to the 3.3.1 version compiled from repository \texttt{CNES/MAJA\_internal}, ran on HAL
		\item MAQT refers here to the Wdark\_fix branch from repository \texttt{Mireille Huc/Chaines\_N2\_N3}, ran on S2CALC
	\end{itemize}
	
	GIPP sources:
	
	\begin{itemize}
		\item MAJA: from repository \texttt{Olivier Hagolle/GIPP\_MAJA}
		\item MAQT: in \texttt{../Exe/PARAM\_N2} from repository \texttt{Mireille Huc/Chaines\_N2\_N3}
	\end{itemize}
\end{frame}

%------------------------------------------------
\begin{frame}
\frametitle{GIPP L2COMM differences}
	
	\begin{block}{diff GIPP\_S2\_MAJA\_3.3\_TM vs GIPP\_S2\_MAJA\_3.3\_TM\_CAMS}
		\begin{tabular}{|l|l|}
			\hline 
			\textit{No CAMS} 					& \textit{CAMS} \\ 
			\hline 
			MT\_Weight\_Threshold: 19 			& MT\_Weight\_Threshold: 10 \\ 
			\hline 
		\end{tabular} 
	\end{block}
	
	\begin{block}{diff GIPP\_S2\_MAJA\_3.3\_TM\_CAMS vs GIPP\_MAQT}
		\begin{tabular}{|l|l|}
			\hline 
			\textit{MAJA} & \textit{MAQT} \\ 
			\hline 
			Cal\_Adjust\_Factor\footnote{\texttt{Cal\_Adjust\_Option=False} in both MAJA and MAQT} : 1.0 1.0 ... 							& Cal\_Adjust\_Factor: 1.0 0.5... \\ 
			\hline 
			Saturation\_Threshold: 1.3 			& Saturation\_Threshold: 0.900 \\ 
			\hline 
			Cirrus\_Alt\_Ref: 2000 				& ? \\ 
			\hline 
			Min\_Threshold\_Var\_Blue: 0.016 	& Min\_Threshold\_Var\_Blue: 0.014 \\ 
			\hline 
			AOT\_KPondCAMS: 0.2 					& KpondCAMS = 0.2 in param\_S2\\ 
			\hline 
			AOT\_HeightScale: 2000 				& ha = 2000 in param\_S2 \\ 
			\hline 
			Cirrus\_Correction\_Option: true 	& Cirrus\_Correction\_Option: false \\ 
			\hline 
			Cirrus\_Max\_Reflectance: 0.07 		& ? \\ 
			\hline 
			Cirrus\_Gamma\_Band\_Codes: B2 B3	& Cirrus\_Gamma\_Band\_Codes: B1 B2 B3 \\ 
			\hline 
		\end{tabular} 
	\end{block}
	
	\textbf{=> GIPP MAJA values used as reference}	

\end{frame}

%------------------------------------------------
\subsection{Test cases}

\begin{frame}
\frametitle{Experimental test cases}
	
	Configurations flags used:
	
	\begin{itemize}
		\item Init (I) or Backward (B) mode, although no explicit Init mode found on MAQT (passing single date \texttt{-d 20180314})
		\item Cirrus correction defaults to True on MAJA while False on MAQT
		\item Environment correction defaults to True (both in MAJA and MAQT)
		\item Use CAMS if True, else Constant\_Model: CONTINEN (both in MAJA and MAQT)
	\end{itemize}
	
	Run naming convention: any combination of \texttt{[MAJA|MAQT]\_[I|B][0|1]\{4\}}, one spare bit left for future addition.	
	
	\begin{center}
		Examples of experimental setup used with both MAJA and MAQT:
		\begin{tabular}{|c|c|c|c|c|c|c|c|c|}
			\hline 
			Test name   & I0000 & I1000 & I1100 & I1110 & B0000 & B1000 & B1100 & B1110\\ 
			\hline 
			Mode        & I     & I     & I     & I     & B     & B     & B     & B   \\ 
			\hline 
			Cirrus Corr & F     & T     & T     & T     & F     & T     & T     & T   \\ 
			\hline 
			Env Corr    & F     & F     & T     & T     & F     & F     & T     & T   \\ 
			\hline 
			CAMS        & F     & F     & F     & T     & F     & F     & F     & T   \\ 
			\hline 
		\end{tabular} 
	\end{center}
	
	
\end{frame}








\begin{frame}
\frametitle{Note on graphics}
	\begin{itemize}
		\item The following scatterplots are resampled outputs of either AOT or SRE
		\item All figures refer to the first image dated back to 2018-03-14
		\item The resampling applied here is 1/20
		\item Edge and cloud masks are used to subset "Land and water" plots (in blue)
		\item The MAQT water mask is added to edge and clouds masks for the "Land only" plots (in green)
		\item On absolute difference maps, a view of the DTM is provided for interpretation of effects of the topography
		\item Units are: AOT (-), SRE (-), altitude (meters a.m.s.l, SRTM based)
	\end{itemize}
\end{frame}




%------------------------------------------------
%------------------------------------------------
\section{MAJA vs MAQT in Init mode, no CAMS, no Cirrus correction}
%------------------------------------------------
%------------------------------------------------



%------------------------------------------------
\subsection{AOT cross-comparison}
\begin{frame}
\frametitle{MAJA vs MAQT in Init mode, no CAMS}
	\begin{center}
		\includegraphics[scale=0.4]{figs/I0000_maja-aot-resampled_vs_I0000_maqt-aot-resampled.png}
	\end{center}
	
	$AOT_{MAJA}$ vs $AOT_{MAQT}$ for all cloud-free pixels (left), and for cloud-free pixels over land only (right), no CAMS, no Cirrus correction (I0000)
\end{frame}

%------------------------------------------------
\begin{frame}
\frametitle{MAJA vs MAQT in Init mode, no CAMS}
	\begin{center}
		\includegraphics[scale=0.48]{figs/diffmap_I0000_maja-I0000_maqt.png}
	\end{center}
	
	Gap-filling implementation differs in MAQT (custom) and MAJA (OTB based)
\end{frame}

%------------------------------------------------
\subsection{SRE cross-comparison}
\begin{frame}
\frametitle{MAJA vs MAQT in Init mode, no CAMS}
	%\begin{columns}
	%\column{.7\textwidth}
		\begin{tabular}{cc}
			\includegraphics[scale=0.2]{figs/I0000_maja-sreB2-resampled_vs_I0000_maqt-sreB2-resampled.png} & \includegraphics[scale=0.2]{figs/I0000_maja-sreB3-resampled_vs_I0000_maqt-sreB3-resampled.png} \\ 
			 
			\includegraphics[scale=0.2]{figs/I0000_maja-sreB4-resampled_vs_I0000_maqt-sreB4-resampled.png} &  \\ 
		\end{tabular} 
		
	%\column{.3\textwidth}
		$SRE_{MAJA}$ vs $SRE_{MAQT}$ for B2 (top left), B3 (top right) and B4 (bottom left)
	%\end{columns}
\end{frame}



%------------------------------------------------
%------------------------------------------------
\section{MAJA vs MAQT in Init mode, with CAMS, no Cirrus correction}
%------------------------------------------------
%------------------------------------------------

%------------------------------------------------


\begin{frame}
\subsection{Bug in MAJA in Init + CAMS mode}
\frametitle{MAJA vs MAQT in Init with CAMS}
	AOT results from MAJA (left) as compared to the expected result produced with MAQT (right) in Init mode with \texttt{Use\_Cams\_Data = true}
	\includegraphics[width=1\textwidth]{figs/intercompInitWithCams.png}
	MAQT: \texttt{/work/CESBIO/projects/Maja/L2A\_MAJA/ArlesIntercomp/31TFJ/MAQT}
	
	MAJA: \texttt{/work/CESBIO/projects/Maja/L2A\_MAJA/ArlesIntercomp/31TFJ/GIPP\_I1110}

\end{frame}

%------------------------------------------------
%------------------------------------------------
\section{MAJA vs MAQT in Backward mode, no CAMS, no Cirrus correction}
%------------------------------------------------
%------------------------------------------------

%------------------------------------------------
\subsection{AOT cross-comparison}
\begin{frame}
\frametitle{MAJA vs MAQT in Backward mode, no CAMS}
	\begin{center}
		\includegraphics[scale=0.4]{figs/B0000_maja-aot-resampled_vs_B0000_maqt-aot-resampled.png}
	\end{center}
	
	$AOT_{MAJA}$ vs $AOT_{MAQT}$ for all cloud-free pixels (left), and for cloud-free pixels over land only (right), no CAMS, no Cirrus correction (I0000)
\end{frame}

%------------------------------------------------
\begin{frame}
\frametitle{MAJA vs MAQT in Backward mode, no CAMS}
	\begin{center}
		\includegraphics[scale=0.48]{figs/diffmap_B0000_maja-B0000_maqt.png}
	\end{center}
\end{frame}

%------------------------------------------------
\begin{frame}
\frametitle{MAJA vs MAQT in Backward mode, no CAMS}
	\begin{center}
		\includegraphics[scale=0.48]{figs/intercompBackwardWithoutCams.png}
	\end{center}
	
	AOT results from MAJA (left) vs MAQT (right) in Backward mode with \texttt{Use\_Cams\_Data = false}
\end{frame}


%------------------------------------------------
\subsection{SRE cross-comparison}
\begin{frame}
\frametitle{MAJA vs MAQT in Init mode, no CAMS}
	%\begin{columns}
	%\column{.7\textwidth}
		\begin{tabular}{cc}
			\includegraphics[scale=0.2]{figs/B0000_maja-sreB2-resampled_vs_B0000_maqt-sreB2-resampled.png} & \includegraphics[scale=0.2]{figs/B0000_maja-sreB3-resampled_vs_B0000_maqt-sreB3-resampled.png} \\ 
			 
			\includegraphics[scale=0.2]{figs/B0000_maja-sreB4-resampled_vs_B0000_maqt-sreB4-resampled.png} &  \\ 
		\end{tabular} 
		
	%\column{.3\textwidth}
		$SRE_{MAJA}$ vs $SRE_{MAQT}$ for B2 (top left), B3 (top right) and B4 (bottom left)
	%\end{columns}
\end{frame}


%------------------------------------------------
\section{What next?}
\begin{frame}
\frametitle{TODO list}
	In Init without CAMS:
	\begin{itemize}
		\item Differences found but limited impact on SRE
		\item Implementation variations for gap-filling, resampling, angles computation for cloud shadows (avg angle in MAQT, one angle per band in MAJA)
		\item Avoid full-resolution cross-comparison
	\end{itemize}
	
	In Init with CAMS: bug in MAJA 3.3.1

	TODO:
	\begin{itemize}
		\item Run/debug MAQT TVA on HAL (clean-up branches in repo)
		\item Bugfix for MAJA in Init + CAMS (FA xxx)
		\item Add another test case with strong AOT (ACIX)?
	\end{itemize}
\end{frame}


%------------------------------------------------
%
%------------------------------------------------

%\begin{frame}
%\frametitle{References}
%\footnotesize{
%\begin{thebibliography}{99} % Beamer does not support BibTeX so references must be inserted manually as below
%\bibitem[Smith, 2012]{p1} John Smith (2012)
%\newblock Title of the publication
%\newblock \emph{Journal Name} 12(3), 45 -- 678.
%\end{thebibliography}
%}
%\end{frame}

%----------------------------------------------------------------------------------------

\end{document}

